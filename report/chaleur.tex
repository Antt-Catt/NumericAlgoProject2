\begin{document}

\subsection{Fonctions principales}
Grâce à la discrétisation numérique, l'équation de la chaleur peut être ramener à la résolution d'un système linéaire. 
La matrice représentant l'opérateur Laplacien étant donné, il a suffit de créer la fonction permettant de la construire.
Cette fonction possède une compléxité en $\Theta(N²)$ étant donné qu'on parcours toute la matrice.\\\\

Ensuite, nous avons mis en place des "situations" où certains points étaient sources de chaleur. Cela à permis de voir comment se comportait
l'algorithme dans plus cas de figures différents. \\\\





\subsection{Fonctions auxiliaires}
Il a été nécessaire de mettre en place des fonctions permettant de passer un vecteur en matrice et inversement.
En effet, pour pouvoir afficher les \emph{Heat maps} il a fallu utiliser des matrices. Cependant, la plupart de nos fonctions renvoyaient
des vecteurs. \\\\

Il a aussi fallu mettre en place une fonction permettant donc d'afficher les \emph{Heat maps}. Pour cela nous avons utilisé matplotlib.
Le module possédant déjà une option pour afficher des gradients de température, il nous a juste fallu s'approprier le code.


\subsection{Résultats}


\end{document}