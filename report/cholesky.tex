\begin{document}

\section{Décomposition de Cholesky}
\label{sec:decomp_cholesky}

L'algorithme de la factorisation dense de Cholesky peut s'écrire ainsi:
\begin{verbatim}

fonction cholesky(A : matrice symétrique définie positive) -> T : matrice triangulaire inférieure
    n = taille(A)
    T = matrice de zéros de taille n x n
    pour i allant de 0 à n-1 faire
        pour j allant de 0 à i-1 faire
            somme = 0
            pour k allant de 0 à j-1 faire
                somme = somme + T[i,k] * T[j,k]
            fin pour
            T[i,j] = (A[i,j] - somme) / T[j,j]
        fin pour
        somme = 0
        pour k allant de 0 à i-1 faire
            somme = somme + T[i,k]^2
        fin pour
        T[i,i] = racine_carree(A[i,i] - somme)
    fin pour
    retourner T
fin fonction

\end{verbatim}

Cet algorithme a une complexité en O($n^3$) avec n la taille de la matrice A prise en paramètre.

Pour résoudre un système linéaire du type A.x=b en utilisant la factorisation de Cholesky, il y a plusieurs étapes : 
\begin{itemize}
    \item La décomposition de Cholesky : on doit calculer la matrice triangulaire inférieure T telle que A=TT^T. Cela peut être fait en utilisant l'algorithme de décomposition de Cholesky décrit précédemment.
    \item La résolution du système triangulaire : on doit résoudre deux systèmes linéaires triangulaires, l'un inférieur T.y=b et l'autre supérieur T^Tx=y. Ces systèmes linéaires triangulaires peuvent être résolus en temps linéaire, car les matrices sont triangulaires. 
\end{itemize}
La complexité de la résolution d'un système triangulaire est en O($n^2$). Ainsi, la complexité totale de la résolution d'un système linéaire dense est: O($n^3$ + $n^2$) soit O($n^3$). 

\subsection{Factorisation complète}
\label{ssec:factor_compl}

Nous allons d'abord écrire un algorithme pour générer des matrices symétriques définies positives creuses avec un nombre de termes extra-diagonaux non nuls réglable. Pour cela, nous allons procéder par étape:
\begin{itemize}
    \item On génère une matrice carrée de taille n avec des termes aléatoires.
    \item On met $n^2$-k termes à 0 (k est le nombre de termes extra-diagonaux non nuls réglable).
    \item On rend la matrice symétrique en effectuant l'opération : A = 0.5(A+$A^{T}$)
    \item On rend la matrice définie postive en calculant : A = A.$A^{T}$.

\end{itemize}

\subsection{Factorisation incomplète}
\label{ssec:factor_incompl}

\end{document}